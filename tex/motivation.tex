\section{
  Motivation
}
%
%==> What is the purpose of git?
%
\begin{frame}
	\frametitle{
	What is the purpose of git?
	}
	\begin{itemize}%[<+->]
	\item
          Suppose you're an exemplary grad student, working {\bf very} diligently on your research.
        \item
          As you're progressing through your project, it is natural for you to experiment.
        \item
          \textcolor{red}{Then you realize, your experiment/the direction you took went awry.}
        \item
          You reach a point where you say \textcolor{blue}{\bf "D\#@\&\$....why did I not save that earlier version of my project!"}
	\end{itemize}
\end{frame}
%
%==> Git comes to the rescue
%
\begin{frame}[fragile]
	\frametitle{
	Git comes to the rescue
	}
	\begin{itemize}%[<+->]
	\item
          \textcolor{red}{
            No more saving {\tt version1.f90, version2.f90, version3.f90, version4.f90, version5.f90,...} of your code.
          }
        \item
          Git will assist in tracking your revisions. 
	\item
          More formally, Git is a \href{
            http://en.wikipedia.org/wiki/Distributed\_revision\_control}{
            \bf distributed version control (DVC)
          }
          and \href{
            http://en.wikipedia.org/wiki/Revision\_control}{
            \bf source code management (SCM)
          } system.
        \item
          More importantly however, \textcolor{blue}{Git is your new best friend!}
	\end{itemize}
\end{frame}
%
%==> The basics of Git
%
\begin{frame}
  \frametitle{
    \bf The basics of Git
  }
  \begin{enumerate}%[<+->]
  \item
    Where can I get Git?
  \item
    Intro and basic commands
  \item
    Remote repositories
  \item
    Resources
  \end{enumerate}
\end{frame}
