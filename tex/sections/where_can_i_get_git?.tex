%
%==> Section 1: Where can I get Git?
%
\section{
  Where can I get Git?
  }
%
%==> Where can I get Git?
%
\begin{frame}[fragile]
  \frametitle{
    Where can I get Git?
  }

  \textcolor{red}{Installing Git:} You can either install it as a package or via another installer, or download the source code and compile it yourself.

\end{frame}

%
%==> Installing on Mac
%
\begin{frame}[fragile]
  \frametitle{
    Installing on Mac
  }
  \begin{itemize}%[<+->]
  \item
    A Mac OS X Git installer is maintained and available for download at \href{http://git-scm.com/download/mac}{http://git-scm.com/download/mac}, or better yet;
  \item
    \textcolor{red}{Homebrew}, “the missing package manager for OS X,” allows you to easily install hundreds of open-source tools, see: \href{https://brew.sh/}{https://brew.sh/}

    \bigskip

    With Homebrew, installing Git is as easy as this:

    \begin{lstlisting}
      brew update
      brew install git
    \end{lstlisting}
    
  \item
    For more detailed instructions, see: \href{http://kj-prince.com/install-git-mac-osx-homebrew/}{http://kj-prince.com/install-git-mac-osx-homebrew/}.
  \end{itemize}

\end{frame}

%
%==> Installing on Windows
%
\begin{frame}[fragile]
\frametitle{\bf
Installing on Windows
}

Windows:\href{http://msysgit.github.io/}{http://msysgit.github.io/}

\end{frame}

%
%==> Installing on GNU/Linux
%
\begin{frame}[fragile]
  \frametitle{
    Installing on GNU/Linux
  }

Use the \textcolor{blue}{package-manager} that comes with your distribution.

\begin{itemize}%[<+->]
\item
  If you're on a Debian-based distribution like Ubuntu, try {\tt apt-get}:
  \begin{lstlisting}
    $ sudo apt-get install git
  \end{lstlisting}
\item
  If you're on Fedora for instance, try {\tt yum}:
  \begin{lstlisting}
    $ sudo yum install git-all
  \end{lstlisting}
\end{itemize}

For different Unix-like flavors, see: \href{http://git-scm.com/download/linux}{http://git-scm.com/download/linux}.

\end{frame}
